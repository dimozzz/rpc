\section{Lower bound}

In this section we will define Delayer strategy that yields lower bound.

\begin{algorithm}
    Input $R \subseteq \{0, 1\}^n \times \{0, 1\}^n$, $A:\{0, 1\}^n \to \{0, 1\}$ and $B:\{0, 1\}^n \to
    \{0, 1\}$.

    Let $\mu(R) = c n$ and $a = \sharp 1_R$.

    \begin{enumerate}
        \item Until $\sigma(R) > \sqrt{n}$.
        \item If there is a $1$-rectangle such that $\mu(R') \ge \frac{c}{8} n$ then:
            \begin{enumerate}
                \item $R \coloneqq R'$.
                \item Goto step 1.
            \end{enumerate}
            
        \item Pick a $0$-rectangle $R'$ such that $\mu(R') \ge \frac{c}{8} n$.
        \item $R \coloneqq R'$.
    \end{enumerate}
\end{algorithm}

\begin{lemma}
    Let $R \subseteq A \times B$ be a rectangle then $\sharp 1_R \le n \min(|R_x|, |R_y|)$.
\end{lemma}

\begin{proof}
    Let $R = X \times Y$, wlog $|X| \le \sigma(R) \le |Y|$. For any $x \in X$ there are at most $n$
    neighbours in the boolean cube.
\end{proof}

\begin{lemma}
    \label{lm:ones-in-small}
    Let $R$ be a rectangle and $\sigma(R) \le 2^{\varepsilon n}$ then $\sharp 1_R \le 2 \varepsilon \sigma(R)$.
\end{lemma}

\begin{proof}
    See stackexchange.
\end{proof}


\begin{lemma}
    Let $R_i$ be a collection of $1$-rectangles, $S = \bigcup R_i$ and $\sharp 1_{S} \ge
    \frac{a}{2}$. There is an $i_0$ such that $\mu(R') \ge \frac{c}{2} n$.
\end{lemma}

\begin{proof}
    $\frac{c}{2} n \sigma(R) = \frac{a}{2} \le \sharp 1_{S} = \sum \sharp 1_{R_i} = \sum \mu(R_i)
    \sigma(R_i) \le \max(\mu(R_i)) \sum \sigma(R_i) \le \max(\mu(R_i)) \sigma(R)$. Hence $\max(\mu(R_i))
    \ge \frac{c}{2}$.
\end{proof}

\begin{lemma}
    Let $R_i$ be a collection of $1$-rectangles, $S = \bigcup R_i$ and $\sharp 1_{S} \ge
    \frac{a}{2}$. There is an $i_0$ such that $\mu(R') \ge \frac{c}{2} n$.
\end{lemma}

\begin{proof}
    $\frac{c}{2} n \sigma(R) = \frac{a}{2} \le \sharp 1_{S} = \sum \sharp 1_{R_i} = \sum \mu(R_i)
    \sigma(R_i) \le \max(\mu(R_i)) \sum \sigma(R_i) \le \max(\mu(R_i)) \sigma(R)$. Hence $\max(\mu(R_i))
    \ge \frac{c}{2}$.
\end{proof}


\begin{lemma}
    Let $R_i$ be a collection of $1$-rectangles, $S = \bigcup R_i$ and $\sharp 1_{S} <
    \frac{a}{2}$. There is a $0$-rectangle such that $\mu(R') \ge \frac{c}{8} n$.
\end{lemma}

\begin{proof}
    There is $0$-rectangle $R'$ that covers at least $\frac{1}{4}$ fraction of ones in zero part of the
    matrix. $\mu(R') = \frac{\sharp 1_{R'}}{\sigma(R')} \ge \frac{\sharp 1_{R}}{2 \sigma(R')} \ge
    \frac{\sharp 1_{R}}{2 \sigma(R)} = \frac{c}{2}$.
\end{proof}


\begin{lemma}
    If $\mu(R) \ge \sqrt{n}$ then $R$ is not a monochromatic.
\end{lemma}

\begin{proof}
    By Lemma \ref{lm:ones-in-small} $\sigma(R) \ge 2^{\frac{\sqrt{n}}{10}}$. $R$ can be $1$-monochromatic
    if $\min(|R_x|, |R_y|) \le n$, but in this case $\sharp 1_{R} \le n^2$ hence $\mu(R) \le
    \frac{n^2}{2^{\frac{\sqrt{n}}{10}}} \ll \sqrt{n}$.

    $R$ is not $0$-monochromatic since $\mu(R) > 0$.
\end{proof}
