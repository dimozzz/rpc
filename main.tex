\documentclass[12pt, fleqn, a4paper]{article}


\usepackage{amsmath}
\usepackage{amssymb}
\usepackage{amsfonts}
\usepackage{textcomp}
\usepackage{amsthm}
\usepackage{xspace}
\usepackage{fullpage}
\usepackage[english]{babel}
\usepackage[utf8]{inputenc}
\usepackage{hyperref}
\usepackage{mathtools}
\usepackage{color}
\usepackage{tikz}
\usepackage{dsfont}

\usetikzlibrary{
	arrows,
	calc,
	patterns,
	intersections,
    backgrounds,
    trees,
    mindmap,
	decorations.pathreplacing,
	decorations.pathmorphing,
	decorations.text,
	decorations.markings,
	shapes,
	positioning
}


\theoremstyle{definition}
\newtheorem{theorem}{Theorem}[section]
\newtheorem*{theorem*}{Theorem}
\newtheorem{lemma}[theorem]{Lemma}
\newtheorem{corollary}[theorem]{Corollary}
\newtheorem{proposition}[theorem]{Proposition}
\newtheorem{fact}[theorem]{Fact}
\newtheorem{problem}[theorem]{Problem}
\newtheorem{exercise}[theorem]{Exercise}
\newtheorem{example}[theorem]{Example}
\newtheorem{definition}[theorem]{Definition}
\newtheorem{remark}[theorem]{Remark}
\newtheorem{algorithm}[theorem]{Algorithm}
\newtheorem{conjecture}[theorem]{Conjecture}

\newcommand{\class}[1]{\mathbf{#1}}
\newcommand{\co}{\mathrm{co}}
\newcommand{\alg}[1]{\mathrm{#1}}
\newcommand{\lang}[1]{\mathtt{#1}}


% classes (1)

\newcommand{\DTime}{\class{DTime}}
\newcommand{\RTime}{\class{RTime}}
\newcommand{\UTime}{\class{UTime}}
\newcommand{\NTime}{\class{NTime}}
\newcommand{\BPTime}{\class{BPTime}}


\renewcommand{\P}{\class{P}}
\newcommand{\ZPP}{\class{ZPP}}
\newcommand{\RP}{\class{RP}}
\newcommand{\coRP}{\co\class{RP}}
\newcommand{\UP}{\class{UP}}
\newcommand{\coUP}{\co\class{UP}}
\newcommand{\NP}{\class{NP}}
\newcommand{\coNP}{{\co}\class{NP}}
\newcommand{\BPP}{\class{BPP}}
\newcommand{\SigmaP}[1]{\Sigma^{#1}\class{P}}
\newcommand{\PH}{\class{PH}}
\newcommand{\PP}{\class{PP}}
\newcommand{\IP}{\class{IP}}
\newcommand{\OP}{\class{\oplus P}}


\newcommand{\EXP}{\class{EXP}}
\newcommand{\MIP}{\class{MIP}}
\newcommand{\NEXP}{\class{NEXP}}
\newcommand{\coNEXP}{{\co}\class{NEXP}}
\newcommand{\MAEXP}{\class{MA}_\class{EXP}}
\newcommand{\D}{\class{D}}
\newcommand{\R}{\class{R}}


% classes (2)

\newcommand{\Ppoly}{\class{P}/\class{poly}}
\newcommand{\NC}{\class{NC}}


\newcommand{\DSpace}{\class{DSpace}}
\newcommand{\NSpace}{\class{NSpace}}
\newcommand{\PSPACE}{\class{PSPACE}}

\newcommand{\EXPSPACE}{\class{EXPSPACE}}



% algorithms and proof systems


\newcommand{\DPLL}{\alg{DPLL}}
\newcommand{\CDCL}{\alg{CDCL}}
\newcommand{\OBDD}{\alg{OBDD}}
\newcommand{\cDC}{c\text{-}\alg{DC}}
\newcommand{\oOBDD}{\text{-}\alg{OBDD}}
\newcommand{\pOBDD}{\pi\oOBDD}
\newcommand{\NBP}{1\text{-}\alg{NBP}}
\newcommand{\DPLLL}{\alg{DPLL}_{lin}}
\newcommand{\ResL}{\alg{Res}_{lin}}
\newcommand{\SemL}{\alg{Sem}_{lin}}
\newcommand{\CP}{\alg{CP}}
\newcommand{\PC}{\alg{PC}}
\newcommand{\PCR}{\alg{PCR}}
\newcommand{\NS}{\alg{NS}}
\newcommand{\LK}{\alg{LK}}
\newcommand{\Th}{\alg{Th}}



% languages

\newcommand{\EQ}{\lang{EQ}}

\newcommand{\SAT}{\lang{SAT}}
\newcommand{\CSP}{\lang{CSP}}
\newcommand{\CSPSAT}{\lang{CSP}\text{-}\lang{SAT}}
\newcommand{\PHP}{\lang{PHP}}
\newcommand{\WPHP}{\lang{WPHP}}
\newcommand{\GNI}{\lang{GNI}}
\newcommand{\MAJSAT}{\lang{MAJ}\text{-}\lang{SAT}}
\newcommand{\QBF}{\lang{QBF}}
\newcommand{\BMS}{\lang{BMS}}
\newcommand{\Bit}{\lang{Bit}}
\newcommand{\MBit}{\lang{MonBit}}
\newcommand{\Search}{\lang{Search}}
\newcommand{\KWm}{\lang{KW}^m}
\newcommand{\Next}{\lang{Next}}
\newcommand{\GPHP}{\lang{G\text{-}PHP}}
\newcommand{\Clique}{\lang{Clique}}
\newcommand{\Peb}{\lang{Peb}}
\newcommand{\PMP}{\lang{PMP}}
\newcommand{\EC}{\lang{EC}}
\newcommand{\TSR}{\lang{TSR}}



\newcommand{\Bad}{Bad}
\newcommand{\OR}{OR}
\newcommand{\AND}{AND}
\newcommand{\Ind}{\lang{Ind}}
\newcommand{\Cl}{\text{Cl}}



% other

\newcommand{\comp}[1][S]{\overline{#1}}
\newcommand{\poly}{\mathrm{poly}}
\newcommand{\Nat}{\mathbb{N}}
\newcommand{\bool}{\{0, 1\}}

\newcommand{\Img}{\mathop{\mathrm{Im}}}

\DeclareMathOperator*{\supp}{supp}
\DeclareMathOperator*{\hd}{hd}
\DeclareMathOperator*{\Exp}{E}
\DeclareMathOperator*{\rk}{rk}


\newcommand{\floor}[1]{\lfloor #1 \rfloor}
\newcommand{\ceil}[1]{\lceil #1 \rceil}

\newcommand{\size}[1]{\mathrm{size}}

\newcommand{\reordering}{\mathrm{reordering}}
\newcommand{\weakening}{\mathrm{weakening}}




\newcommand{\set}[2][ ]{\{#2 \ifthenelse{\equal{#1}{ }}{ }{~|~#1}\}}


\DeclareRobustCommand{\rchi}{{\mathpalette\irchi\relax}}
\newcommand{\irchi}[2]{\raisebox{\depth}{$#1\chi$}}

\newcommand{\comment}[1]{}

\newcommand{\seepage}[2][See]{
    \marginnote{
        \scriptsize {#1} p.~\pageref{#2}
    }
}

\newenvironment{reusable}[2]{
	\ifcsname c@#2_help\endcsname
    \else
	    \expandafter\newcounter{#2_help}
	\fi

    \restatable{#1}{#2}
    \expandafter\label{#2}
    \expandafter\ifnum\value{#2_help} = 0
    	\seepage[For proof see]{#2_app}
    \else
        \seepage{#2}
    \fi
}{
    \endrestatable
}

\newcommand{\reuse}[1]{
	\expandafter\stepcounter{#1_help}
    \expandafter\label{#1_app}
    \csname#1\endcsname*
}

\newcommand{\chone}{\mathds{1}}


\title{
    $\RPC$
}

\author{
}


\begin{document}
%	\maketitle

    \section{Preliminaries}


    Let $A, B \subseteq \{0, 1\}^{n \times b}$ be sets that correspond to inputs of Alice and Bob.

Let $U_{A, r} \subseteq \binom{n}{r} \times \{0, 1\}^r$ be a set of all projections of set $A$ on some
$r$ coordinates. By analogy we can define $V_{B, r}$.

\begin{definition}
\label{def:graphs}
    Let $r \le n$ be an integer, $C = \{0, 1\}^{n \times b}$, $f_A: A \to \{0, 1\}^{n \times b}$ and
    $f_B: B \to \{0, 1\}^{n \times b}$. A weighted underected graph $(U_{A, r}, C)$ contain edge $(u, c)$
    with weight $w$ iff $|{a \in A \mid a \sim u, f_A(a) = c}| = w$. By analogy we can define
    $(C, V_{B, r})$. 
\end{definition}

\section{Lower bound}

In this section we will define Delayer strategy that yields lower bound.

\begin{algorithm}
    Input $A, B \subseteq \{0, 1\}^{n \times b}$, $f_A: A \to \{0, 1\}^{n \times b}$ and $f_B: B \to
    \{0, 1\}^{n \times b}$. $(U_{A, r}, C)$ and $(C, V_{B, r})$ be graphs from definition
    \ref{def:graphs}.

    \begin{enumerate}
        \item Until ????!!!!
        \item $Purify(A, B, r)$.
        \item If there is a vertex $u \in U_{A, r}$ and edge $(u, c)$ such that $deg(u) \le 2 * w(u, c)$
            and $deg(v) \le 2 * w(c, v)$, where $v$ is a reflection of $u$ then:
            \begin{enumerate}
                \item Delayer answer is one.
                \item $A \coloneqq \{a \subseteq A \mid a \sim u, f_A(a) = c\}$.
                \item $B \coloneqq \{b \subseteq B \mid b \sim v, f_B(b) = c\}$.
                \item $r \coloneqq \frac{r}{2}$.
                \item Goto step 1.
            \end{enumerate}
        \item If there is a vertex $u \in U_{A, r}$ and edge $(u, c)$ such that $deg(u) \le 2 * w(u, c)$
            then:
            \begin{enumerate}
                \item Delayer answer is zero.
                \item $A \coloneqq \{a \subseteq A \mid a \sim u, f_A(a) = c\}$.
                \item $B \coloneqq \{b \subseteq B \mid b \sim v, f_B(b) \neq c\}$.
                \item $r \coloneqq \frac{r}{2}$.
                \item Goto step 1.
            \end{enumerate}
        \item Delayer answer is zero.
        \item Choose a partition of $C$ into two parts $C'$ and $C''$ such that avegare degree in induced
            subgraphs $(U_{A, r}, C')$ and $(V_{B, r}, C'')$ be at least $0.1 \cdot d_{A, r}^{avg}$.
        \item $A \coloneqq \{a \subseteq A \mid a \sim u, f_A(a) \in C'\}$.
        \item $B \coloneqq \{b \subseteq B \mid b \sim v, f_B(b) \in C''\}$.
        \item $r \coloneqq \frac{r}{2}$.
    \end{enumerate}
\end{algorithm}
%    \section{Simulation}

In this section we want to generalize lifting theorem \ref{!!!!}.

\begin{definition}
    We say that a function $f(x, y)$ is $\sigma$-\textit{good} iff communication matrix of this function
    contains a monochromatic rectangle of uniform measure $\sigma$. 
\end{definition}


\begin{theorem}
    Let $f:\{0, 1\}^n \to \{0, 1\}$ be an arbitrary function and $\S$ be a class of $\sigma$-good
    functions and $g$ satisfy $(\delta, h)$-h.r.d. then $\D^{\S}(f \circ g) \ge \Omega(\sigma \D_{dt}(f))$.
    % тут нужны граничные условия всякие.... да и вообще паршиво сформулировано, но без доказательства
    % тяжело константы давать
\end{theorem}

To prove this Theorem we need the following lemmas.

\begin{lemma}[\cite{GPW!!!, !!!!}]
    For any $p \ge 2$, if $A \subseteq X^p$ is $\varphi$-avg.-thick then for every $\delta \in (0, 1)$
    there is a $\frac{\delta}{p} \varphi$-thick subset $A' \subseteq A$ with $|A'| \ge (1 - \delta)
    |A|$.
\end{lemma}


\begin{lemma}[\cite{!!!!}]
    Let $h \ge 1$, $p \ge 2$ and $i \in [p]$ be integers and $\delta, \tau, \varphi \in (0, 1)$ be reals,
    where $\tau \ge 2^{-h}$. Consider a function $g: X \times Y \to \{0, 1\}$ which has $(\delta,
    h)$-h.r.d. Suppose $A \times B \subseteq X^p \times Y^p$ is a nonempty $\tau$-thick rectangle, and
    suppose also that $d_{avg}(A, i) \le \varphi |X|$. Then for any $c \in \{0, 1\}$
\end{lemma}
    
	\bibliographystyle{alpha}
    \bibliography{main}

\end{document}
